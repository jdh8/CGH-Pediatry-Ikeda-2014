\documentclass{beamer}
\usepackage[no-math]{fontspec}
\usepackage{xeCJK}
\setCJKmainfont{Source Han Sans TW}
\hypersetup{colorlinks,linkcolor=}

\usetheme{CambridgeUS}
\title[(Ikeda \textit{et al}, 2018)]{
    Changes in the perfusion waveform of the internal cerebral vein and
    intraventricular hemorrhage in the acute management of extremely
    low-birth-weight infants
}
\subtitle{Ikeda \textit{et al}, \textit{Eur J Pediatr}, 2014}
\author[Chen-Pang He]{何震邦 (Chen-Pang He), Intern}
\date{December 12, 2018}
\institute[CGH]{Cathay General Hospital}

\newcommand*{\solo}[1]{\centering\includegraphics[width=\textwidth, height=0.8\textheight, keepaspectratio]{#1}}

\begin{document}
\maketitle

\section{Introduction}
\begin{frame}{Introduction}
    \begin{itemize}
        \item Intraventricular hemorrhage is hemorrhage of the internal cerebral veins.
        \item It is a major complication with both acute and long-term effects.
        \item It affects the neurological prognoses of extremely low-birth-weight infants.
    \end{itemize}
\end{frame}

\section{Methods}
\begin{frame}{Measurement of internal cerebral vein perfusion}
    \begin{itemize}
        \item Measurement of ICV perfusion by Doppler ultrasound
        \item Approach via the anterior fontanelle
    \end{itemize}
\end{frame}

\begin{frame}{Figure 1}
    \solo{F1.png}
\end{frame}

\begin{frame}{Classification of the perfusion waveform}
    \begin{description}
        \item[Grade 0] Constant speed
        \item[Grade 1] Minimum speed > half maximum
        \item[Grade 2] 0 < minimum speed < half maximum
        \item[Grade 3] Minimum speed = 0
    \end{description}
\end{frame}

\begin{frame}{Figure 2}
    \solo{F2.png}
\end{frame}

\begin{frame}{Study design}
    \begin{itemize}
        \item Prospective observational study
        \item Single center
            \begin{itemize}
                \item Aomori Prefectural Central Hospital
            \end{itemize}
    \end{itemize}
\end{frame}

\begin{frame}{Figure 3}
    \solo{F3.png}
\end{frame}

\section{Results}
\begin{frame}{Figure 4}
    \solo{F4.png}
\end{frame}

\begin{frame}{Table 1}
    \solo{T1.eps}
\end{frame}

\begin{frame}{Table 2}
    \solo{T2.eps}
\end{frame}

\begin{frame}{Table 3}
    \solo{T3.eps}
\end{frame}

\begin{frame}{Table 4}
    \solo{T4.eps}
\end{frame}

\begin{frame}{Table 5}
    \solo{T5.eps}
\end{frame}

\section{Discussion}
\begin{frame}{Discussion}
    \begin{itemize}
        \item Previous studies have shown that the incidence of IVH increases with
            \begin{itemize}
                \item Lower gestational age and birth weight
                \item Low Apgar score, male sex, chorioamnionitis, non-administration of maternal steroids
            \end{itemize}
        \item This study shows correlation between the incidence of IVH and
            fluctuations in the perfusion waveform of the internal cerebral
            veins.
    \end{itemize}
\end{frame}

\begin{frame}{Present study}
    \begin{itemize}
        \item Perinatal factors and fluctuations of the perfusion waveform of ICV were investigated.
        \item Gestational age is negatively correlated to IVH incidence.
        \item IVH incidence is positively correlated to high-grade fluctuations.
            \begin{itemize}
                \item Gestational age is also negatively correlated to high-grade fluctuations.
                \item Data from infants of gestational age < 26 weeks were analyzed and gave consistent results.
            \end{itemize}
    \end{itemize}
\end{frame}

\begin{frame}{Onset and disappearance of fluctuations}
    \begin{itemize}
        \item In most cases, the perfusion waveform of ICV was steady within 3
            h after birth, with grade 1 fluctuations evident in only 2 cases.
        \item Both these infants were small for the gestational age and had
            been delivered early because of fetal heart failure, and their
            fluctuations disappeared at 16 h after birth.
        \item The majority of the observed fluctuations first occurred at least
            24 h after birth and disappeared at 120 h to 144 h after birth.
    \end{itemize}
\end{frame}

\begin{frame}{Onset and disappearance of high-grade fluctuations}
    \begin{itemize}
        \item In 24 of 25 cases (96\%) with high-grade fluctuations, the
            high-grade fluctuations were found for the first time within 72 h
            of birth.
        \item In this study, the changes in the perfusion waveform also tended
            to become strong from 24 to 72 h after birth and then disappeared
            at 96 h after birth.
    \end{itemize}
\end{frame}

\begin{frame}{IVH cases}
    \begin{itemize}
        \item There were 8 cases of IVH in this study.
        \item The only case of IVH in the low-grade group occurred within 24 h
            after birth.
        \item The other 7 cases occurred from 24 to 96 h after birth.
            \begin{itemize}
                \item They showed high-grade fluctuations before IVH occurred.
            \end{itemize}
    \end{itemize}
\end{frame}

\begin{frame}{Limitations}
    \begin{itemize}
        \item Since the present study included a small number of IVH patients,
            the statistical analysis was insufficient to analyze the changes in
            the fluctuations of the perfusion waveform according to time and
            its association with IVH.
        \item There is no report on central blood flow, and details of the
            ductus arteriosus or other central cardiovascular issues are
            lacking in this study.
    \end{itemize}
\end{frame}
\end{document}
